\chapter{Basissoftware}


\section{Schichtenarchitektur}

Die in AUTOSAR definierte Schichtenarchitektur soll eine einfachere Portierung von Software auf unterschiedliche Hardware ermöglichen. Mussten bislang bei ungünstig konzipierten Software-Architekturen verschiedene Stellen bis hin zur Anwendungsschicht umfangreich angepasst werden, müssen mit AUTOSAR lediglich alle mikrocontroller-spezifischen Treiber im MCAL ersetzt werden. Dadurch reduziert sich der Implementierungs- und Testaufwand sowie das damit verbundene Risiko deutlich. Softwarekomponenten können so durch die hardwareunabhängige Schnittstellen leicht auf unterschiedliche Steuergeräte übertragen werden.
In AUTOSAR unterscheidet man grundsätzlich die folgende Schichten:

\begin{itemize}
\item Anwendungsschicht
\item Runtime Environment (RTE)
\item Service Layer
\item ECU Abstraction Layer
\item Microscontroller Abstraction Layer (MCAL)
\end{itemize}


Die Umsetzung des Projekts erforderte konkrete Implementierungen sowohl im Bereich der Service Layer als auch in der ECU Hardware Abstraction.

%Allgemeiner Aufbau des Schichtenmodells

\subsection{Service Layer}

Die Service Layer enthält im Allgemeinen die Betriebsystem-Funktionen und stellt verschiedene Arten von Hintergrunddiensten wie Netzwerkdienste, Speicherverwaltung und Buskommunikationsdienste für die Anwendungsschicht bereit. Sie ist die höchste Schicht der Basissoftware und ist damit essentiell für die Anwendungsschicht. Die Implementierung ist in den meisten Fällen hardwareunabhängig und damit leicht austauschbar.




% Beschreibung Aufgaben der Service Layer



\subsubsection{Sound Handler}

Der Sound Handler stellt der RTE spezifische Funktionen zur Audiowiedergabe zur Verfügung.  
Als Funktionalität implementiert der Sound Handler dabei sowohl die einfache Wiedergabe eines Tons oder einer WAV-Datei als auch die Mehrfachwiedergabe mit optionaler Pause. Umgesetzt wird dies durch die folgenden Methoden:

\begin{lstlisting}[frame=single]  
void play_single_wav(U32 freq, U32 ms, U32 vol);
void play_single_wav(U32 file, U32 length, U32 freq, U32 vol);

void play_multiple_tones(U32 freq, U32 ms, U32 vol, U32 rep, U32 pause);
void play_multiple_wavs(const CHAR *file, U32 length, S32 freq, U32 vol, U32 rep, U32 pause);
\end{lstlisting}

% Beschreibung der aufrufbaren Funktionen von RTE und MCAL, Methodendeklaration

\subsubsection{Motor Handler}

Der Motor Handler erlaubt die Ansteuerung der Motoren. 

%TODO

\begin{lstlisting}[frame=single]  
void motor_set_speed(S8 speed_left, S8 speed_right);

S8 motor_read_speed();
\end{lstlisting}


% Beschreibung der aufrufbaren Funktionen von RTE und MCAL, Methodendeklaration

\subsubsection{Communication Handler}

Der Communication Handler ist für die Bluetooth Verbindung zwischwen zwei Geräten zuständig. Da eine Bluetooth-Verbindung eine Master-Slave Verbindung ist, muss die physische Slave-Adresse im Code definiert werden.
Diese wurde mithilfe eines simplen konstangten Byte-Array gelöst. Für die RTE wird eine initialize, send, recv und terminate Funktion zur Verfügung gestellt, wobei die send und recv Funktion mit einem Ringbuffer arbeiten und somit asynchron sind.
Um Threadsafety zu gewährleisten, werden in diesen beiden Funktionen für das Lesen und Schreiben des Ringbuffers alle Interrupts mithilfe der Funktion \textit{DisableAllInterrupts} deaktiviert und bei Verlassen der Funktion mithilfe von \textit{EnableAllInterrupts} wieder aktiviert.
Es gibt 2 Tasks, einen für send und einen für recv, welche dann die Ringbuffer befüllen oder auslesesen. Diese Tasks sind in der OIL-Datei mit \textit{SCHEDULE = NON} deklariert, damit diese nicht unterbrochen werden können und weiterhin Threadsafety ohne Implementierung von einem Mutex oder Semaphor gewährleistet ist.
Damit Pakete nicht auseinander gerissen werden, muss ein einzelnes receive-Runnable im Application Layer definiert werden, welches dann die Zusammensetzung für empfangene Daten zu einem ganzen Paket übernimmt und die anderen Runnables benachrichtigt.

\begin{lstlisting}[frame=single]  
//Returns 1 on success, 0 on failure
U8 com_init(U8 is_master);
//Returns number of sent bytes, 0 on error [Async]		
U32 com_send(U8 *buff, U32 len);
//Returns number of received bytes, 0 on error [Async]	
U32 com_recv(U8 *buff, U32 len);	
void com_terminate();
\end{lstlisting}

% Beschreibung der aufrufbaren Funktionen von RTE und MCAL, Methodendeklaration, Tasks, Events, Counter

\subsubsection{Display Handler}
Der Display Handler bietet Funkionalität zum Darstellen von Textausgaben auf dem Display des Bricks an.
Ganze Strings und einzelne Zeichen können entweder fortlaufend oder an einer bestimmten Stelle des Monitors angezeigt werden.
Mit der Funktion \textit{display\_clear\_line} können ganze Zeilen gelöscht werden. Folgende Funktionen stehen zur Verfügung:

\begin{lstlisting}[frame=single]  
void display_write_xy(int x, int y, const char *str);
void display_clear_line(int y);
void display_write(const char *str);
\end{lstlisting}


\subsection{ECU Hardware Abstraction}

Die ECU Hardware Abstraction Layer versteckt den konkreten Aufbau des Steuergeräts und bietet einen einheitlichen Zugriff auf alle Funktionalitäten eines Steuergeräts wie Kommunikation, Speicher oder IO - unabhängig davon, ob diese Funktionalitäten Bestandteil des Mikrocontrollers sind oder durch Peripheriekomponenten realisiert werden.

% Beschreibung Aufgaben der ECU Abstraction Layer

\subsubsection{Ultraschall Hardware Abstraction}

% Beschreibung der aufrufbaren Funktionen von RTE und MCAL, Methodendeklaration

\subsubsection{ADC Hardware Abstraction}

% Beschreibung der aufrufbaren Funktionen von RTE und MCAL, Methodendeklaration, ADC Interfaces, Zusammenspiel mit I2C
\subsubsection{DIO Hardware Abstraction} \label{dio}

Die DIO Hardware Abstraction stellt der RTE ein Interface zur Ansteuerung externen digitalen Ein- und Ausgänge zur Verfügung. Da der NXT Brick keinen interne DIO Einheit besitzt wird hier nur das Interface für die externe DIO Ansteuerung genutzt.

\begin{lstlisting}[frame=single,caption={DIO Interface},captionpos=b]  
#define DIO_Read_Data(DIOIndex, Port, Adresse) \
	DioIfReadFctPtr[DIOIndex](Port, Adresse)
	
#define DIO_Write_Data(DIOIndex, Port, Adresse, Data) \
	DioIfWriteFctPtr[DIOIndex](Port, Adresse, Data)
\end{lstlisting}
Generell ist das Interface für die Ansteuerung aller digitalen IO-Ports konzipiert worden.
Der Funktionen \textit{Read} und \textit{Write} wird ein Index übergeben. Dieser verweist in einem Array auf die dazugehörige interne oder externe Funktion.\\
Um die Taster des Roboters auszulesen, muss somit ein Funktionsaufruf der \textit{Read}-Funktion mit dem Index 0 getätigt werden.

\begin{lstlisting}[frame=single,caption={Aufruf der internen DIO-Read Funktion},captionpos=b]  
U8 dio_read_int(U8 port_id, U8 i2c address){
	return ecrobot_get_touch_sensor(port_id);
}
\end{lstlisting}
Das Ansteuern der LEDs geschieht über den Funktionsaufruf \textit{Write}. Hier muss der Index 1 übergeben werden, da sich die LEDs an einem externen IO-Port befinden. Da auch die ADC Hardware Abstraction mit dem I$^2$C Expander arbeitet, wird an dieser Stelle der Zugriff über die I$^2$C Hardware Abstraction erfolgen. Dieser ist in Kapitel \ref{i2cabstraction} beschrieben.

\begin{lstlisting}[frame=single,caption={Aufruf der I$^2$C Hardware Abstraction},captionpos=b]  
void dio_write_ext(U8 port_id, U8 i2c address, U8 data){
	i2c_write(port_id, i2c_address, data);
}
\end{lstlisting}
 
% Beschreibung der aufrufbaren Funktionen von RTE und MCAL, Methodendeklaration, DIO Interfaces, Zusammenspiel mit I2C

\subsubsection{I$^2$C Hardware Abstraction} \label{i2cabstraction}

Die I$^2$C Hardware Abstraction ermöglicht den Zugriff auf einen externen I$^2$C-Expander. Hierfür kann z.B. über die bereits beschriebene DIO Hardware Abstraction \ref{dio} die Funktion \textit{i2c\_read} oder \textit{i2c\_write}
aufgerufen werden.\\
\begin{lstlisting}[frame=single,caption={Aufruf der \textit{i2c\_write}-Funktion},captionpos=b]

void i2c_write(U8 port_id, U8 i2c_address, U8 data){
	
	ecrobot_send_i2c(port_id,i2c_address,data,&data,sizeof(data));
}
\end{lstlisting}
Die Funktion erhält als Übergabeparameter einen Port, an dem der Expander am Roboter angeschlossen ist, eine Adresse, mit der der richtige IC angesprochen werden kann und ein Datum, dass gesendet werden soll.\\
\begin{lstlisting}[frame=single,caption={Aufruf der \textit{i2c\_read}-Funktion},captionpos=b]  
U8 i2c_read(U8 port_id, U8 i2c_address){
	
	U8 buffer;
	ecrobot_read_i2c(port_id, i2c_address, buffer, &buffer, sizeof(buffer));
	return buffer;
}
\end{lstlisting}
Die Funktion erhält als Übergabeparameter einen Port, an dem der Expander am Roboter angeschlossen ist und eine Adresse, mit der der richtige IC angesprochen werden kann. Als Rückgabewert wird das Datum, das gelesen wurde zurückgegeben.


% Beschreibung des I2C Aufbaus, Methodenaufrufe